% Preamble for pmath347_notes.tex

\usepackage[dvipsnames, table]{xcolor}

\usepackage{amsmath}
\usepackage{amssymb}
\usepackage{amsthm}
\usepackage{changepage}
\usepackage{enumitem}
\usepackage{fancyhdr}
\usepackage{forest}
\usepackage{fullpage}
\usepackage{geometry}
\usepackage{graphicx}
%\usepackage{bbold}
\usepackage{mathrsfs}
\usepackage{mathtools}
\usepackage{multicol}
\usepackage{multirow}
\usepackage{parskip}
\usepackage[notmath]{sansmathfonts}
\usepackage{stmaryrd}
\usepackage{tabularx}
\usepackage[most]{tcolorbox}
\usepackage{tikz}
\usepackage{titlesec}
\usepackage{titletoc}
\usepackage[titles]{tocloft}
\usepackage[normalem]{ulem}

\usepackage[
  pdftitle={PMATH 347 Notes},
  pdfsubject={University of Waterloo, Spring 2021 (William Slofstra)},
  pdfauthor={Marco Yang <kc4yang@uwaterloo.ca>},
  colorlinks=true,
  linkcolor=blue
]{hyperref}
\usepackage[nameinlink]{cleveref}

\usetikzlibrary{arrows.meta}

%% Layout

\graphicspath{{./img/}}

% Margins
\geometry{
  margin=1in,
  headheight=1ex + \baselineskip,
  headsep=\baselineskip
}

% Header/footer
\pagestyle{fancy}
\fancyhf{}
\renewcommand{\headrulewidth}{0pt}
\renewcommand{\sectionmark}[1]{\markboth{\thesection\hspace{1.5ex}{#1}}{}}
\fancyhead[L]{\color{black!50} \small \sffamily PMATH 347}
\fancyhead[R]{\color{black!50} \small \sffamily \leftmark}
\fancyfoot[C]{\color{black!50} \small \sffamily \thepage}

% Chapters
\counterwithout*{section}{chapter} % Don't reset section number in new chapter
\titleformat{\chapter}
  {\huge \sffamily \bfseries \centering} % Format
  {Week \thechapter:} % Label
  {1ex} % Sep
  {} % Before
\titlecontents{chapter}
  [0em] % Left spacing
  {\vspace{\baselineskip}} % Above
  {\large \bfseries \contentslabel{2em}} % Numbered format
  {\large \bfseries} % Numberless format
  {\hfill} % Filler page format
  [\vspace{\baselineskip}] % Below

% Sections
\renewcommand{\thesection}{\arabic{section}}
\newcommand{\sectionbreak}{\clearpage\phantomsection}
\titleformat{\section}
  {\Large \sffamily \bfseries} % Format
  {\thesection:} % Label
  {1ex} % Sep
  {} % Before
\titlecontents{section}
  [3em] % Left spacing
  {} % Above
  {\bfseries \contentslabel{2em}} % Numbered format
  {} % Numberless format
  {\titlerule*{$\cdot$}\contentspage} % Filler page format
  [] % Below

% Subsections
\titleformat{\subsection}
  {\large \sffamily \bfseries} % Format
  {} % Label
  {0pt} % Sep
  {} % Before
\titlecontents{subsection}
  [3em] % Left spacing
  {} % Above
  {} % Numbered format
  {} % Numberless format
  {\hfill} % Filler page format
  [] % Below

%% Commands

% New week
\newcommand{\week}[1]{
  \chapter{#1}
  \thispagestyle{empty}
  \pagebreak
}

% Environments
\newtcbtheorem[no counter]
  {thm} % environment name
  {Theorem} % display name
  { % options
    colback=blue!10,
    colframe=blue!10,
    colbacktitle=blue!10,
    coltitle=blue!60!black,
    fonttitle=\sffamily\bfseries,
    sharp corners,
    boxsep=1ex,
    toptitle=1ex,
    before skip=\baselineskip,
    after skip=\baselineskip,
    separator sign={~---},
    label type=thm
  }
  {thm} % label prefix
\crefname{thm}{Theorem}{Theorems}

\newtcbtheorem[no counter]
  {lem} % environment name
  {Lemma} % display name
  { % options
    colback=blue!10,
    colframe=blue!10,
    colbacktitle=blue!10,
    coltitle=blue!60!black,
    fonttitle=\sffamily\bfseries,
    sharp corners,
    boxsep=1ex,
    toptitle=1ex,
    before skip=\baselineskip,
    after skip=\baselineskip,
    separator sign={~---},
    label type=lem
  }
  {lem} % label prefix
\crefname{lem}{Lemma}{Lemmas}

\newtcbtheorem[no counter]
  {cor} % environment name
  {Corollary} % display name
  { % options
    colback=blue!10,
    colframe=blue!10,
    colbacktitle=blue!10,
    coltitle=blue!60!black,
    fonttitle=\sffamily\bfseries,
    sharp corners,
    boxsep=1ex,
    toptitle=1ex,
    before skip=\baselineskip,
    after skip=\baselineskip,
    separator sign={~---},
    label type={cor}
  }
  {cor} % label prefix
\crefname{cor}{Corollary}{Corollaries}

\newtcbtheorem[no counter]
  {prop} % environment name
  {Proposition} % display name
  { % options
    colback=blue!10,
    colframe=blue!10,
    colbacktitle=blue!10,
    coltitle=blue!60!black,
    fonttitle=\sffamily\bfseries,
    sharp corners,
    boxsep=1ex,
    toptitle=1ex,
    before skip=\baselineskip,
    after skip=\baselineskip,
    separator sign={~---},
    label type={prop}
  }
  {prop} % label prefix
\crefname{prop}{Proposition}{Propositions}

\newtcbtheorem[no counter]
  {exer} % environment name
  {Exercise} % display name
  { % options
    colback=red!10,
    colframe=red!10,
    colbacktitle=red!10,
    coltitle=red!60!black,
    fonttitle=\sffamily\bfseries,
    sharp corners,
    boxsep=1ex,
    toptitle=1ex,
    before skip=\baselineskip,
    after skip=\baselineskip,
    separator sign={~---},
    label type={exer}
  }
  {exer} % label prefix
\crefname{exer}{Exercise}{Exercises}

\newtcbtheorem[no counter]
  {defn} % environment name
  {Definition} % display name
  { % options
    parbox=false,
    nameref/.style={},
    colback=green!10,
    colframe=green!10,
    colbacktitle=green!10,
    coltitle=green!60!black,
    fonttitle=\sffamily\bfseries,
    sharp corners,
    boxsep=1ex,
    toptitle=1ex,
    before skip=\baselineskip,
    after skip=\baselineskip,
    separator sign={~---},
    label type={defn}
  }
  {defn} % label prefix
\crefname{defn}{Definition}{Definitions}

\newtcolorbox{ex}[1][Example]{
  enhanced,
  parbox=false,
  sharp corners,
  breakable,
  boxrule=0pt,
  left=1ex + 2mm + 4pt,
  right=0pt,
  bottom=0pt,
  frame hidden,
  title={#1},
  fonttitle=\sffamily\bfseries,
  colback=white,
  coltitle=red!60!black,
  colbacktitle=white,
  borderline west={4pt}{0pt}{red!60!black},
  before skip=\baselineskip,
  after skip=\baselineskip
}

\makeatletter
\newenvironment{proofb}{%
  \par
  \pushQED{\qed}
  \normalfont \topsep0\p@\@plus6\p@\relax
  \trivlist
  \item[]\ignorespaces
}{%
  \popQED\endtrivlist\@endpefalse
}
\makeatother

\newenvironment{thmproof}[1][Proof.]{
  \begin{tcolorbox}[
    enhanced,
    breakable,
    parbox=false,
    sharp corners,
    boxrule=0pt,
    left=1ex + 2mm + 4pt,
    right=0pt,
    bottom=0pt,
    frame hidden,
    title={#1},
    fonttitle=\sffamily\itshape,
    colback=white,
    coltitle=blue!60!black,
    colbacktitle=white,
    borderline west={4pt}{0pt}{blue!60!black},
    before skip=\baselineskip,
    after skip=\baselineskip
  ]
  \begin{proofb}
}{
  \end{proofb}
  \end{tcolorbox}
}

\newenvironment{exerproof}[1][Proof.]{
  \begin{tcolorbox}[
    enhanced,
    breakable,
    parbox=false,
    sharp corners,
    boxrule=0pt,
    left=1ex + 2mm + 4pt,
    right=0pt,
    bottom=0pt,
    frame hidden,
    title={#1},
    fonttitle=\sffamily\itshape,
    colback=white,
    coltitle=red!60!black,
    colbacktitle=white,
    borderline west={4pt}{0pt}{red!60!black},
    before skip=\baselineskip,
    after skip=\baselineskip
  ]
  \begin{proofb}
}{
  \end{proofb}
  \end{tcolorbox}
}

% Emphasis
\newcommand{\hldef}[1]{\textcolor{green!60!black}{\textbf{#1}}}

% Circled numbers
\newcommand*\circled[1]{
  \tikz[baseline=(char.base)]{
    \node[shape=circle, draw, inner sep=2pt] (char) {\footnotesize #1};
  }
}

% Enum with circled numbers
\newenvironment{enumcase}[1][]{
  \begin{enumerate}[label=\protect\circled{\arabic*}, #1]
}{
  \end{enumerate}
}

%% Math commands

% Useful delimiters: abs
\DeclarePairedDelimiter\abs{\lvert}{\rvert}
\DeclarePairedDelimiter\norm{\lVert}{\rVert}
\DeclarePairedDelimiter\ceil{\lceil}{\rceil}
\DeclarePairedDelimiter\floor{\lfloor}{\rfloor}
\DeclarePairedDelimiter\ang{\langle}{\rangle}

% Operators and such
\DeclareMathOperator{\Id}{Id}
\DeclareMathOperator{\GL}{GL}
\DeclareMathOperator{\PGL}{PGL}
\DeclareMathOperator{\SL}{SL}
\DeclareMathOperator{\Fun}{Fun}
\DeclareMathOperator{\Hom}{Hom}
\DeclareMathOperator{\Sub}{Sub}
\DeclareMathOperator{\supp}{supp}
\let\Im\relax
\DeclareMathOperator{\Im}{Im}
\DeclareMathOperator{\Conj}{Conj}

% Macros
\newcommand{\Zmod}[1]{\mathbb{Z}/{#1}\mathbb{Z}}
\newcommand{\orbit}{\mathcal{O}}
